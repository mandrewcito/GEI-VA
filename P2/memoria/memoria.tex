% Usar el tipo de documento: Artículo científico.
\documentclass[12pt,a4paper]{article}

% Cargar mensajes en español.
\usepackage[spanish]{babel}

% Usar codificación utf-8 para acentos y otros.
\usepackage[utf8]{inputenc}

%Dimensiones de los márgenes.
\usepackage[margin=1.5cm]{geometry}

% Insertar porciones de código
\usepackage{listings}

% Comenzar párrafos con separación no indentación.
\usepackage{parskip}

% Usar gráficos
\usepackage{graphicx}
\usepackage{caption}
\usepackage{subcaption}
%
% Usar contenedores flotantes para figuras.
\usepackage{float}

% Carpeta de las imágenes.
\graphicspath{{img/}}

% Configuración para porciones de código.
\lstset{
%	language=bash,
	basicstyle=\ttfamily\small,
%	numberstyle=\footnotesize,
%	numbers=left,
%	backgroundcolor=\color{gray!10},
%	frame=single,
	tabsize=4,
%	rulecolor=\color{black!30},
%	title=\lstname,
%	escapeinside={\%*}{*)},
	breaklines=true,
	breakatwhitespace=true,
%	framextopmargin=2pt,
%	framexbottommargin=2pt,
	extendedchars=false,
	inputencoding=utf8
}
%%%%%%%%%%%%%%%%%%%%%%%%%%%%%%%%%%%%%%%%%%%%%%%%%%%%%%%%%%%%%%%%%%%%%%%%%%%%%%%

% Propiedades
\title{Biometría Caras}

\author{Andrés Baamonde Lozano (andres.baamonde@udc.es)}

\begin{document}

\maketitle

%%%%%%%%%%%%%%%%%%%%%%%%%%%%%%%%%%%%%%%%%%%%%%%%%%%%%%%%%%%%%%%%%%%%%%%%%%%%%%%
\clearpage 

\tableofcontents

\clearpage 

\section{Introducción}
Esta práctica consiste en aplicar técnicas de visión artificial para ser capaz de identificar personas
mediante el análisis de caras. Se dispone para cada individuo de una serie de imágenes tomadas en
instantes diferentes.
\section{Procedimiento}
\subsection{Preprocesado de la imagen}
En este caso, primeramente pasaremos las imagenes a jpg para que openCV pueda procesarlas.Cuando ya podamos procesar la imagen, comenzaremos con el preprocesado de la imagen que constaría principalmente de la ecualización del histograma o alguna clase de umbralizado pero descartamos estas técnicas debido a que el histograma de la imagen tiene toda la información en los niveles de gris más altos.
\subsection{Detección,delimitación de la cara y de sus zonas relevantes}
Para detectar y delimitar la cara y sus zonas relevantes utilizaremos las Haar Cascades que son las características de imagen digitales utilizadas en el reconocimiento de objetos. Deben su nombre a su similitud intuitiva con wavelets de Haar y se utilizaron en el primer detector de caras en tiempo real. Viola y Jones adaptaron la idea y desarrollaron las Haar-like.Que por ejemplo estas utilizan características del tipo la zona de las mejillas es mas clara que la de los ojos considerando un rectángulo central y los adyacentes. Por lo tanto una característica haar común para la detección de rostros es un conjunto de dos rectángulos adyacentes que se encuentran por encima del ojo y la región de la mejilla. La posición de estos rectángulos se define en relación a una ventana de detección que actúa como una caja de delimitación al objeto de destino.
\begin{figure}[H]
\centering
\includegraphics[scale=0.5]{deteccion.png}
\caption{Detección de cara, ojos, nariz y boca}
\end{figure}


\subsection{Cálculo de propiedades}
En las propiedades de la imagen utilizaremos SIFT, y un método de cálculo de áreas que trataremos más adelante.
\begin{figure}[H]
\centering
\includegraphics[scale=0.5]{P2Siftpy.png}
\caption{SIFT a cada cara}
\end{figure}
\subsection{Obtención del patrón biométrico}
El patrón biométrico serán, los puntos de interés, que dependiendo de la aproximación, estarán organizados de diferente forma refinandolos para que así sean más relevantes.
\subsection{Comparación de imágenes}
En cuanto al método para reconocer las caras seran comparándolas y haciendo un matching de las características anteriormente calculadas.También se podría utilizar un vector en el que con varias imágenes se hallara un vector de características más robusto, pero, en este caso no disponemos de ejemplos suficientes.
\section{Aproximaciones}
\subsection{Comparación de el área entre regiones}
En esta aproximación veremos como funciona la comparación de el area de dos regiones sobre la cara. Una región sera la que delimitan los ojos y la nariz, y otra región será la que delimitan los ojos y la boca.
\subsection{Puntos característicos por fuerza bruta}
En esta aproximación lo que se hace es aplicar sift a las dos imágenes para luego matchearlos con otra imagen con un knn, con la primera se entrena y con la segunda se hace el test, utilizamos los keyPoints que obtenemos del algoritmo.
\begin{figure}[H]
\centering
\includegraphics[scale=0.7]{matches.png}
\caption{SIFT fuerza bruta}
\end{figure}
\subsection{Puntos característicos en la cara}
En esta aproximación hacemos lo mismo que en la anterior,pero tan sólo en la cara.
\begin{figure}[H]
\centering
\includegraphics[scale=0.7]{matchesCara.png}
\caption{SIFT sólo cara}
\end{figure}
\subsection{Puntos característicos por regiones de la cara}
En esta aproximación hacemos lo mismo que en la anterior, pero dividiendo la cara en cuatro zonas para así asegurarnos de un matching más preciso.
\begin{figure}[H]
\centering
\includegraphics[scale=0.7]{matchesCaraDivision.png}
\caption{SIFT dividiendo la cara en cuatro regiones}
\end{figure}
\subsection{Puntos característicos por regiones características}
En esta aproximación hacemos el matching pero solo de las zonas características de la cara humana, ojos, nariz y boca.
\begin{figure}[H]
\centering
\includegraphics[scale=0.7]{matchesCaraCaracteristicas.png}
\caption{SIFT en ojos,boca,nariz}
\end{figure}
\end{document}